\chapter{はじめに}
近年のAIの発展は目覚ましく、画像分類や異常検知(参考文献追加)などの単純なルールで記述する事が困難なタスクや、更には長らく人間に固有の技術であると考えられてきた画像や文章の生成の分野においてさえ、高い性能を発揮するまでに
AI技術は成長した。
特に直近3年のstable diffusion\cite{diffusion}・ Instruct GPT\cite{GPT}の登場は専門家間に留まらず一般社会に大きな影響を与え、多くの人々が人間とAIの将来的な関係性について考えざるを得ないフェーズに入ったと言える。(ここも参考文献追加?)
この「優れたAIに対して人間はどう接するべきか」という命題を考える際には既に人間を大きく凌駕したAIが存在する領域において手法の構築や実験を行うのが適当である。
そのため、本論文では囲碁・将棋・チェスなどの主要なボードゲームにおいて人間を大きく凌駕したパフォーマンスを誇るAlphaZero\cite{AlphaZero}を題材に、説明性付与を試みる。

しかし、優れたAIが社会で広く実用化され、受容されるためにはAIの判断や生成物が生み出される過程の透明性、信頼性の担保が必要不可欠である。
説明性が必要とされるのは医療や金融などの判断の慎重さが求められるべき分野のみではない。画像生成・文章生成AIが上述の様に一度オープンソース化された2024年現在においてもG7広島サミットにおいて
国際的な協調の観点からAIの透明性を確保する重要性が指摘されている\cite{Hiroshima}。また、日本国内においても「信頼できるAI」を実現する必要性が認識されている\cite{グランドデザイン}。
それに加えて、将来的に「優れたAIに学ぶ」という探求心や学習意欲からも説明性へのニーズが広く湧き起こる事が予測される。
現に囲碁・将棋・チェスなどの主要なボードゲームをオンラインでプレイできるサービスにはゲーム終了後の振り返り(感想戦)においてAIの判断やAIによる想定図を閲覧できるサービスが数多く提供されている。(ここに参考文献を入れる)
研究においてもボードゲームの学習支援という形で「AIに学ぶ」試みが存在する。しかし、後述するようにそれらの研究は各ゲームのドメイン知識等を使用したものが多い。

そこで本論文ではなるべくドメイン知識を用いず、AIの判断を可視化する手法を考案しその有効性を検証した。
手法の詳細は第3章に記載しており、実験結果は第4章に記載している。
