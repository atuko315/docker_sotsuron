\chapter{はじめに}
近年のAIの発展は目覚ましく,画像分類や異常検知などの単純なルールで記述する事が困難なタスクや、更には長らく人間に固有の技術であると考えられてきた画像や文章の生成の分野においてさえ、高い性能を発揮するまでに
AI技術は成長した。
特に昨年のstable diffusion\cite{diffusion}, Instruct GPT\cite{GPT}の登場により人間の生産労働の在り方, 人間とAIの関係,ひいては
この先の社会がAIとどう付き合っていくのか,AIによってどう変わっていくのかを専門家だけでなく一般の人々も含めて考えざるを得ない段階に差し掛かっていると言える。
この「優れたAIに対して人間はどう接するべきか」という命題を考える際には既に人間を大きく凌駕したAIが存在する領域において手法の構築や実験を行うのが適当である。
そのため、ここではconnect4と呼ばれる比較的単純なボードゲームを題材とし、2016年に当時世界有数のプレイヤーであったイ・セドルを四勝一敗で圧倒したAlphaGo\cite{AlphaGo}\cite{Nikkei}を簡易的に模したネットワークであるAlphaZero\_baseline\cite{baseline}を用いて本論文を執筆した。
優れたAIが社会で広く実用化され、受容されるためにはAIの判断や生成物(以下単純に「出力」という表現を使用する)が生み出される過程の透明性がいずれは必要不可欠になることが予測される。
実際に画像生成AIが上述の様に一度オープンソース化された2024年現在においても新技術の国際的な協調の観点からG7広島サミットにおいてAIの透明性を確保する重要性が指摘されており\cite{Hiroshima}、日本国内においても「信頼できるAI」を実現する必要性が認識されている\cite{グランドデザイン}。
また、そのような法的・倫理的観点によるAIの透明性への希求だけでなく「どうすれば人間もAIのような成果を生むことができるのか」という探求心や学習意欲から成る説明性へのニーズが広く湧き起こる事が予測される。
特にゲームのように「AI対人間」と表現すべき対立構造を強く有する領域においては後者のニーズがより大きくなっていくものと予測される。
現に囲碁,将棋,チェスなどの主要なボードゲームをオンラインでプレイできるサービスにはゲーム終了後の振り返り(感想戦)においてAIの判断やAIによる想定図を閲覧できるサービスが数多く提供されている。
研究においてもボードゲームの学習支援という形で「AIに学ぶ」試みが存在する。しかし、後述するようにそれらの研究は各ゲームのドメイン知識等を使用した、従来の人間による指導方法の自動化に近い形態のものが多い。
ここでは人間の学習のサポートツールとしてAIを用いるというよりは、AIの動作を人間のユーザー向けに説明する手法を構築し、人間にAIの挙動を理解してもらうことを通じて人間側のスキルを向上させる試みを行った。
本論文における提案はゲームの勝敗を左右する地点を検出する指標の定義、そして適切な予測を決定木から取り出す手法の二つに大別される。
また、実験も提案手法を用いてゲームの終了状態を予測するデータ実験と、ユーザーの使用感や学習効果を調査するシステム実験の二つを行い、提案手法の有効性を検証した。