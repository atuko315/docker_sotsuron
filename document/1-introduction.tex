\chapter{はじめに}
近年のAI分野の発展は目覚ましく、画像分類や異常検知(参考文献追加)などの単純なルールで記述する事が困難なタスクや、更には長らく人間に固有の技術であると考えられてきた画像や文章の生成の分野においてさえ、高い性能を発揮している。
特に直近3年のStable Diffusion\cite{diffusion}・ Instruct GPT\cite{GPT}の登場により、専門家でない人々がAIを使用する機会が増加した。
そのため、人間とAIの将来的な関係性に対する考察の必要性が高まっているといえる。。(ここも参考文献追加?)
この「優れたAIに対して人間はどう接するべきか」という命題を考える際には、既にスピードや能力において人間を大きく凌駕したAIを題材とし、手法の構築や実験を行うのが適当である。
そのため、本論文では囲碁・将棋・チェスなどの主要なボードゲームにおいて人間の実力を凌駕したパフォーマンスを誇るAlphaZero\cite{AlphaZero}を題材に、説明性付与を試みる。

しかし、優れたAIが社会で広く実用化され受容されるためには、AIの出力が生み出される過程の透明性、信頼性の担保が必要不可欠である。
説明性が必要とされるのは医療や金融などの判断の慎重さが求められるべき分野のみではない。上述の様に2024年現在、画像生成・文章生成AIはオープンソース化され、広く普及した。
そのような状況においても、G7広島サミットでは国際的な協調の観点からAIの透明性を確保する重要性が指摘されている\cite{Hiroshima}。
また、日本国内においても「信頼できるAI」を実現する必要性が認識されている\cite{グランドデザイン}。
それに加えて、将来的には「優れたAIに学ぶ」という探求心や学習意欲からも説明性へのニーズが広く湧き起こる事が予測される。
現に囲碁・将棋・チェスなどの主要なボードゲームをオンラインでプレイできるサービスには、ゲーム終了後の振り返り(感想戦)においてAIの判断やAIによる進行図を閲覧できるサービスが数多く提供されている。(ここに参考文献を入れる)
研究においてもボードゲームの学習支援という形で「AIに学ぶ」試みが存在する。しかし、後述するようにそれらの研究は各ゲームに特有の知識(ドメイン知識)に依存するものが多い。

そこで本論文ではなるべくドメイン知識を用いず、AIの判断を可視化する手法を考案しその有効性を検証する。

続く第2章では関連研究について記載し、第3章では提案手法の詳細を述べ、第4章に行った2つの実験の結果と考察を記す。

