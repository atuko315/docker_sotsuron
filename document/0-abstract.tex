\abstract
2024年現在,チェス,囲碁,将棋などのメジャーな2人用ボードゲームにおいてAIは人間を遥かに凌駕するようになった\cite{Nikkei}\cite{deepBlue}\cite{dennou}.
しかし,AlphaZero\cite{AlphaZero}の登場から5年が経過した今もなおその十分な説明手法は登場していない.
そこで,AIによる類似した予測図を複数提示することでAIの判断根拠の可視化を試みた.
題材として対戦型ボードゲームの1つであるconnect4\cite{connect4}を選択した.
実験はAI同士の対戦データを用いてゲームの終了状態を予測するデータ実験と,提案手法がタスクの上達度や楽しさに与える効果を検証するインタフェース実験の2種類を行った.
結果としてデータ実験では提案手法が比較手法よりも高い予測精度を示し,インタフェース実験では提示する情報量が極端に大きいまたは小さい場合に,提案手法は比較手法よりも被験者が「AIの意図を掴めた」
と感じやすいという結果が出た.