\abstract
2024年現在,チェス,囲碁,将棋などのメジャーな2人用ボードゲームにおいてAIは人間を遥かに凌駕するようになった\cite{Nikkei}\cite{deepBlue}\cite{dennou}.
しかし,AlphaZero\cite{AlphaZero}の登場から5年が経過した今もなおその十分な説明手法は登場していない.
そこで,AIによる類似した予測図を複数提示することでAIの判断根拠の可視化を試みた.
題材として対戦型ボードゲームの1つであるconnect4\cite{connect4}を選択した.
実験はAI同士の対戦データを用いてゲームの結果を予測するデータ実験と,提案手法がタスクの上達度に与える効果を検証するインタフェース実験の2種類を行った.
結果としてデータ実験では提案手法が比較手法よりも高い予測精度を示し,インタフェース実験では提示する情報量が極端に大きいまたは小さい場合に,提案手法は比較手法よりも被験者が「AIの意図を掴めた」
と感じやすいという結果が出た.