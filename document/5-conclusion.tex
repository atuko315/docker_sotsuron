\chapter{結論}

本論文では
\begin{itemize}
	\item 決定木からの有力な分岐を抽出
	\item 分岐を結果の類似性によりグループ化
\end{itemize}

する事により,強化学習AIシステムの判断の可視化を試みた.
提案手法の有効性を示すため,connect4(ボードゲーム)を題材として2つの実験を行った.
第1の実験ではAI同士の対戦データを用いた提案手法のゲームの終了状態に対する予測の妥当性を評価し,特定の指標において比較手法に対して高い予測精度を示した.
第2の実験では自作のGUIシステムを用いた提案手法の学習支援効果を検証し,ユーザーに提示される情報量が極端に大きい場合または小さい場合に,比較手法よりも高いユーザー評価
を得うることがわかった.

課題としては,第4章において提案手法をconnect4に適用する際に「石を4つ並べたプレイヤーが勝利する」というゲームのルールに基づき
「4つ並んだ石の座標」を用いてノードのグループ化を行っていることが挙げられる.
この事はアルゴリズムの一部でドメイン知識を用いている事を意味し,さらなる汎用化の余地が存在する.
また,ユーザーに提示する情報の量や種類もさらなる調査の必要性を感じた.
そのため,
今後は
\begin{itemize}
	\item 手法のゲーム固有の知識(ドメイン知識)からの脱却
	\item ユーザーの学習支援に適切な情報提示の方式の模索
\end{itemize}


に取り組みたい.
