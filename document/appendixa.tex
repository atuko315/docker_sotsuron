\chapter{データ実験の詳細}
\section{使用したモデルの詳細}
第三章で述べたとおり使用した対戦データは弱いAIを先番とし、強いAIを後手としている。
AIの強さは一手ごとの探索を行う時間(time)と付録Cで述べる$C_{puct}$の値によって調整した。
timeと$C_{puct}$はいずれも値が大きい程モデルは強くなると考えられる。
対戦データ生成時のパラメータは以下の表の幅からゲームごとにランダムな値を採用した。
これはパラメータの値を変化させることでゲームデータに多様性を持たせるためである。
\begin{table}[H]
	\caption{対戦データのパラメータ}
	\centering
	\scalebox{0.98}[0.98]{
		\begin{tabular}{c|c|c}
			モデル&強&弱\\
			time    & 3-5 & 0-2 \\ \hline
			$C_{puct}$ & 0.8-1   & 0-0.5 \\

		\end{tabular}
	}
	\label{table:battle}
\end{table}

\section{対戦結果の詳細}
2000ゲームのうち1983ゲームは強いAIの勝利となった。
またゲームごとの手数は75%のゲームが36手以内で終了している。



\section{ヒストリカルデータ}
為