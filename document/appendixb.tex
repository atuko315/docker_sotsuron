\chapter{各種アルゴリズムの詳細}
\label{chap:alg}
\section{PV-モンテカルロ木探索}
PV-モンテカルロ木探索の疑似コードをAlgorithm\ref{alg:mcts-1},Algorithm\ref{alg:mcts-2}に示す.
\begin{algorithm}
    \caption{PV-MCTS in AlphaZero (Part 1: Exploration)}
    \label{alg:mcts-1}
    \begin{algorithmic}[1]
        \State $t$: 決定木
        \State $T$: 遷移関数
        \State $N(s, a)$: $(s, a)$の組み合わせを探索した回数
        \State $Q(s, a)$: 行動価値関数 
        \State $W(s, a)$: 行動価値の総和$(W(s, a)=Q(s, a)N(s, a))$
        \State $P(s, a)(=P(s_n), s_n=T(s, a))$: 
        \State ニューラルネットワークから出力された方策
        \State $V(s, a)(=V(s_n), s_n=T(s, a))$: 
        \State ニューラルネットワークから出力された局面評価
        \Function{Explore}{$s_{\text{start}}$}
            \State Set $s_{now} = s_{start}$ and $a_{now} = a_m$
            \For {each simulation}
                \State $\zeta \gets \textrm{empty list}$
                \State $s_{\text{current}} \gets s_{\text{start}}$
                \While {$s_{\text{current}}$ がゲームの終了状態でない場合}
                    \State $a_t \gets TreePolicy(s)$
                    \State (Tは遷移関数)
                    \State $(s_{\text{current}}, a_t)$を $\zeta $の末尾に追加
                    \State $s_{\text{next}} \gets T(s_{\text{current}}, a_t)$
                    \State $s_{\text{current}} \gets s_{\text{next}}$
                    
                \EndWhile
                \State $G \gets V(s_e)$ 
                \State($s_e$は$s_{start}$から探索してたどり着いたノード$s_e$)
                \State \Call{Backpropagate}{$\zeta, G$}
            \EndFor
        \EndFunction
        
        
    \end{algorithmic}
\end{algorithm}
\begin{algorithm}
    \caption{PV-MCTS in AlphaZero (Part 2: Backpropagation)}
    \label{alg:mcts-2}
    \begin{algorithmic}[1]
        \Function{TreePolicy}{$s$}
            \If {$s$が探索されていない子ノードを持つとき}
                \State $s_c \gets T(s, a)$ ($s_c$は未探索のノード)
                \State \Call{InitNode}{$s_c$}
                \State \Return $a$
            \Else
                \State 以下のPUCTスコア$U(s, a)$を計算
                \State $U(s, a)= C(s)P(s, a)\frac{\sqrt{N(s)}}{1+N(s, a)}$
                \State $C(s)=\textrm{log}\frac{1+N(s)+C_{\textrm{base}}}{C_{\textrm{base}}}+C_{\textrm{init}}$
                \State($C_{\textrm{base}}$, $C_{\textrm{init}}$はハイパーパラメータ)
                \State $(N(s)=\Gamma N(s, a))$
                \State 以下のように$a$を求める
                \State $a = {\textrm{argmax}}_a (Q(s, a)+U(s, a))$
                \State \Return $a$
                
            \EndIf
        \EndFunction
        \Function{Backpropagate}{$\zeta, G$}
            \For {each node-action pair $(s, a)$ in $\zeta$}
                \State $N(s, a) \gets  N(s, a)+1$
                \State $W(s, a) \gets  W(s, a)+G$
                \State $Q(s, a) \gets \frac{W(s, a)}{N(s, a)}$
            \EndFor
        \EndFunction
        \Function{InitNode}{$s$}
            \For {each action $a$ from  $s$}
                \State $N(s, a) \gets 0$
                \State $W(s, a) \gets 0$
                \State $Q(s, a) \gets 0$
            \EndFor
        \EndFunction
    \end{algorithmic}
\end{algorithm}

\section{提案手法のアルゴリズム}
提案手法のアルゴリズムの疑似コードをAlgorithm\ref{alg:myalg-1},Algorithm\ref{alg:myalg-2}に示す.
\begin{algorithm}
    
    \caption{提案手法のアルゴリズム(part1)}
    \label{alg:myalg-1}
    \begin{algorithmic}[1]
        \State $t$: 手法を適用する探索木
        \State $T$: 状態遷移関数
        \State $l$: 盤面の収集を行う手数
        \State $k$: 一度に集める盤面の数
        \State $\zeta(s, s')$: ノード$s$から$s'$までの軌跡
        
       
        \Function{MyAlgorithm}{$s_{start}, a, l, k$}
           \State $s_{now}\gets s_{start}$
           \State $a_{now}\gets a$
           \State $Z_0 \gets$\Call{CollectBoards}{$s_{now}, a_{now}, l, k$}
           \State $(Z_0 =\{\zeta(s_{start}, {s'}_1), ..., \zeta(s_{start}, {s'}_{k^l})\})$
           \State $ Z \gets$ empty list
           \For {each $\zeta(s_{start}, {s'}_i)$ in $Z_0$}
             \State $\zeta(s_{start}, s_{{edge}_i}) \gets$ \Call{Traverse}{${s'}_i, \zeta(s_{start}, {s'}_i)$}
             \State $\zeta(s_{start}, s_{{edge}_i})$を$Z$の末尾に追加
           \EndFor
           \State 収集された終了状態の集合$S(=\{s_{edge_1},s_{edge_2}, ..., s_{edge_{k^l}}\})$を任意の共通項$c$で副集合$\{S_1, S_2, ..., S_q\}$に分割する
           \State 最も要素数の多い副集合$S_{max}$中の各要素$\{s_{e_1}, s_{e_2}, ...,  s_{e_u}\}$
           \State に対応する軌跡の集合$Z_{max}(=\{\zeta(s_{start}, s_{e_1}), \zeta(s_{start}, s_{e_2}), ..., \zeta(s_{start}, s_{e_u})\})$を保存
           \State \Return $Z_{max}$
        \EndFunction
    \end{algorithmic}
\end{algorithm}
\begin{algorithm}
    
    \caption{提案手法のアルゴリズム(part2)}
    \label{alg:myalg-2}
    \begin{algorithmic}[1]    
    \scriptsize   
        \Function{CollectBoards}{$s, a, l, k$}
           \State $s_{now} \gets T(s, a)$
           
           \State $Z \gets $ empty queue
            \If{$s$が未探索のノード($N(s)=0$)または終了状態のとき}
             \Return  $Z$
            \EndIf
           \State 訪問回数$N(s)$から上位$k$の行動$\{a_1, a_2, ..., a_{k}\}(=\alpha)$を取り出す
           \For {each $a_i$ in $\alpha$}
             
             \State $s_{{next}_i} \gets T(s_{now}, a_i)$
             \State $\zeta(s_{now},s_{{next}_i})(=\{s_{now}, a_i, s_{{next}_i}\})$を$Z$の末尾に追加
           \EndFor
           \If{l=1}
             \Return $Z$
           \EndIf
           \State $i \gets 1$
           \While {$i < l$}
                \For {each $\zeta(s_{now}, s_{j})$ in $Z$}
                    \State $\zeta(s_{now}, s_{j})$ を $Z$からポップ
                    \If{$s$が未探索のノード($N(s)=0$)または終了状態のとき}
                        \State $\zeta(s_{now}, s_{j})$ を$k$回$Z$の末尾に追加
                        \State continue
                    \EndIf
                    \State 訪問回数$N(s)$から上位$k$の行動$\{a_1, a_2, ..., a_{k}\}(=\alpha)$を取り出す
                    \For {each $a_i$ in $\alpha$}
                        \State $s_{{next}_j} \gets T(s_{j}, a_i)$
                        \State $\zeta(s_{now},s_{{next}_i})(=\zeta(s_{now}, s_{j}).append({a_i, s_{{next}_i}}))$を$Z$の末尾に追加
                    \EndFor
                    
                \EndFor     
           \EndWhile
           \Return $Z(=\{\zeta(s_{start}, {s'}_1), ..., \zeta(s_{start}, {s'}_{k^l})\})$
        \EndFunction
        \Function {Traverse}{$s, \zeta(s_{start}, s)$}
        \State $s_{now} \gets s$
        \State $\zeta_r \gets \zeta(s_{start}, s)$
        \While {$s_{now}が探索済みかつ終了状態でない$}
            \State $a_t \gets \textrm{argmax}_a N(s_{now}, a)$
            \State $s_n \gets T(s_{now}, a_t)$
            \State $\zeta_r.append({a_t, s_n})$
            \State $s_{now} \gets s_n$
        \EndWhile
        \Return $\zeta_r$
        \EndFunction
       
        
    \end{algorithmic}
\end{algorithm}

\section{比較手法のアルゴリズム}
比較手法の疑似コードをAlgorithm \ref{alg:compare}に示す.疑似コード中のTraverse()は Algorithm\ref{alg:myalg-2}と同一である.
\begin{algorithm}
    \caption{比較手法のアルゴリズム}
    \label{alg:compare}
    \begin{algorithmic}[1]       
        \State $t$: 手法を適用する探索木
        \State $T$: 状態遷移関数
        \State $\zeta(s, s')$: ノード$s$から$s'$までの軌跡
        \Function{CompareAlgorithm}{$s, a$}
           \State $s_n \gets T(s, a)$
           \State $\zeta(s, s_n) \gets \{s, a, s_n\}$
           \State $\zeta \gets$ \Call{$s_n, \zeta(s, s_n)$}
           \Return $\zeta$
        \EndFunction
        
        \Function {Traverse}{$s, \zeta(s_{start}, s)$}
            \State $s_{now} \gets s$
            \State $\zeta_r \gets \zeta(s_{start}, s)$
            \While {$s_{now}が探索済みかつ終了状態でない$}
                \State $a_t \gets \textrm{argmax}_a N(s_{now}, a)$
                \State $s_n \gets T(s_{now}, a_t)$
                \State $\zeta_r.append({a_t, s_n})$
                \State $s_{now} \gets s_n$
            \EndWhile
            \Return $\zeta_r$
        \EndFunction
       
        
    \end{algorithmic}
\end{algorithm}


\section{提案手法におけるニューロ補間}
第4章におけるデータ実験,インタフェース実験ではいずれも手法のニューラルネットワークによる補間を行っている.
第4章で示した疑似コードでは未探索のノードにたどり着いた際は走査を終了する.ニューラルネットワークによる補間とはこの場合,
方策$P(s, a)$で訪問回数$N(s, a)$を代用し走査を継続することである.
Algorithm\ref{alg:neuro-1},Algorithm\ref{alg:neuro-2}にニューロ補間を行う場合の疑似コードを示す(変更部分に下線).
比較手法に対してニューロ補間を行う場合も同様にTraverse()のコードを変更する.
\begin{algorithm}
    \caption{提案手法のアルゴリズム(ニューロ補間あり)part1}
    \label{alg:neuro-1}
    \small
    \begin{algorithmic}[1]       
        \Function{CollectBoards}{$s, a, l, k$}
           \State $s_{now} \gets T(s, a)$
           \State $Z \gets $ empty queue
            \If{$s$が終了状態のとき}
             \Return  $Z$
            \EndIf
            \If{$s$が未探索のとき}
             \State \underline{方策$P(s)$から上位$k$の行動$\{a_1, a_2, ..., a_{k}\}(=\alpha)$を取り出す}
            \Else
             \State 訪問回数$N(s)$から上位$k$の行動$\{a_1, a_2, ..., a_{k}\}(=\alpha)$を取り出す
            \EndIf
           \For {each $a_i$ in $\alpha$}
             \State $s_{{next}_i} \gets T(s_{now}, a_i)$
             \State $\zeta(s_{now},s_{{next}_i})(=\{s_{now}, a_i, s_{{next}_i}\})$を$Z$の末尾に追加
           \EndFor
           \If{l=1}
             \Return $Z$
           \EndIf
           \State $i \gets 1$
           \While {$i < l$}
                \For {each $\zeta(s_{now}, s_{j})$ in $Z$}
                    \State $\zeta(s_{now}, s_{j})$ を $Z$からポップ
                    \If{$s$終了状態のとき}
                        \State $\zeta(s_{now}, s_{j})$ を$k$回$Z$の末尾に追加
                        \State continue
                    \EndIf
                    \If{$s$が未探索のとき}
                       \State \underline{方策$P(s)$から上位$k$の行動$\{a_1, a_2, ..., a_{k}\}(=\alpha)$を取り出す}
                   \Else
                    \State 訪問回数$N(s)$から上位$k$の行動$\{a_1, a_2, ..., a_{k}\}(=\alpha)$を取り出す
                   \EndIf
                    
                    \For {each $a_i$ in $\alpha$}
                        \State $s_{{next}_j} \gets T(s_{j}, a_i)$
                        \State $\zeta(s_{now},s_{{next}_i})(=\zeta(s_{now}, s_{j}).append({a_i, s_{{next}_i}}))$を$Z$の末尾に追加
                    \EndFor
                    
                \EndFor     
           \EndWhile
           \Return $Z(=\{\zeta(s_{start}, {s'}_1), ..., \zeta(s_{start}, {s'}_{k^l})\})$
        \EndFunction
    \end{algorithmic}
\end{algorithm}
\begin{algorithm}
    \caption{提案手法のアルゴリズム(ニューロ補間あり)part2}
    \label{alg:neuro-2}
    \begin{algorithmic}[1] 
        \Function {Traverse}{$s, \zeta(s_{start}, s)$}
        \State $s_{now} \gets s$
        \State $\zeta_r \gets \zeta(s_{start}, s)$
        \While {$s_{now}$が終了状態でない}
            \If{$s_{now}$が未探索のとき}
                \State \underline{$a_t \gets \textrm{argmax}_a P(s_{now}, a)$}
            \Else
                \State $a_t \gets \textrm{argmax}_a N(s_{now}, a)$
            \EndIf
            \State $s_n \gets T(s_{now}, a_t)$
            \State $\zeta_r.append({a_t, s_n})$
            \State $s_{now} \gets s_n$
        \EndWhile
        \Return $\zeta_r$
        \EndFunction
       
        
    \end{algorithmic}
\end{algorithm}
\section{インタフェース実験における提案手法の変更}
\label{sec:fix}
インタフェース実験では画面の左下に盤面$s$の訪問回数$N(s)$と局面評価$V(s, a)$を表示する.このとき
$V(s, a)$の絶対値(どちらのプレイヤーの勝利に近いか)と予想図の勝敗が異なる場合,ユーザーの混乱を招く可能性がある.
そのため,提案手法では勝敗が$V(s, a)$の絶対値と一致する軌跡を表示するように修正を施した.
インタフェース実験用に修正された疑似コードはAlgorithm:\ref{alg:system}となる.(下線が変更部分)
また,提案手法により収集した最終盤面$S=\{s_{edge_1}, s_{edge_2}, ..., s_{edge_{k^l}}\}$は
最終的に4つ以上連続してつながっている石の組み合わせ(fatal group)でグループ化し,最も要素数が多い集合の軌跡をユーザーに提示した.

\begin{algorithm}
    \caption{提案手法のアルゴリズム(インタフェース実験)}
    \label{alg:system}
    \begin{algorithmic}[1]       
        
        \Function {Traverse}{$s, \zeta(s_{start}, s)$}
        \State  \underline{$\zeta(s_{start}, s)$から$s_{start}$の次の行動$a$を取り出す}
        \State \underline{$v \gets V(s, a)$}
        \State $s_{now} \gets s$
        \State $\zeta_r \gets \zeta(s_{start}, s)$
        \While {$s_{now}$が終了状態でない}
            \If{$s_{now}$が未探索のとき}
                \State \underline{$a_t \gets \textrm{argmax}_a P(s_{now}, a)$}
            \Else
                \State $a_t \gets \textrm{argmax}_a N(s_{now}, a)$
            \EndIf
            \State $s_n \gets T(s_{now}, a_t)$
            \State $\zeta_r.append({a_t, s_n})$
            \State $s_{now} \gets s_n$
        \EndWhile
        \If{\underline{$v$と$V(s_{now})$の絶対値が異なるとき}}
           \Return null
        \EndIf
        \Return $\zeta_r$
        \EndFunction
       
        
    \end{algorithmic}
\end{algorithm}
