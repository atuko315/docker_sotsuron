\chapter{各種アルゴリズムの詳細}
\section{比較手法のアルゴリズム}
比較手法の疑似コードを以下に示す。疑似コード中のTraverse()は第三章の疑似コード \ref{alg:myalg-1} \ref{alg:myalg-2}と同一である。
\begin{algorithm}
    \caption{比較手法のアルゴリズム}
    \begin{algorithmic}[1]       
        \State $t$: 手法を適用する探索木
        \State $T$: 状態遷移関数
        \State $\zeta(s, s')$: ノード$s$から$s'$までの軌道
        \Function{CompareAlgorithm}{$s, a$}
           \State $s_n \gets T(s, a)$
           \State $\zeta(s, s_n) \gets {s, a, s_n}$
           \State $\zeta \gets$ \Call{$s_n, \zeta(s, s_n)$}
           \Return $\zeta$
        \EndFunction
        
        \Function {Traverse}{$s, \zeta(s_{start}, s)$}
            \State $s_{now} \gets s$
            \State $\zeta_r \gets \zeta(s_{start}, s)$
            \While {$s_{now}が探索済みかつ終了状態でない$}
                \State $a_t \gets \textrm{argmax}_a N(s_{now}, a)$
                \State $s_n \gets T(s_{now}, a_t)$
                \State $\zeta_r.append({a_t, s_n})$
                \State $s_{now} \gets s_n$
            \EndWhile
            \Return $\zeta_r$
        \EndFunction
       
        
    \end{algorithmic}
\end{algorithm}


\section{提案手法におけるニューロ補間}
第四章におけるデータ実験、システム実験ではいずれも手法のニューラルネットワークによる補間を行っている。
第四章で示した疑似コードでは未探索のノードにたどり着いた際は走査を終了する。ニューラルネットワークによる補間とはこの場合、
方策$P(s, a)$で訪問回数$N(s, a)$を代用し走査を継続することである。
以下にニューロ補間を行う場合の疑似コードを示す(変更部分に下線)。
比較手法に対してニューロ補間を行う場合も同様にTraverse()のコードを変更する。
\begin{algorithm}
    \caption{提案手法のアルゴリズム(ニューロ補間あり)}
    \begin{algorithmic}[1]       
        \Function{CollectBoards}{$s, a, l, k$}
           \State $s_{now} \gets T(s, a)$
           
           \State $Z \gets $ empty queue
            \If{$s$が終了状態のとき}
             \Return  $Z$
            \EndIf
            \If{$s$が未探索のとき}
             \State \underline{方策$P(s)$から上位$k$の行動${a_0, a_1, ..., a_{k-1}}(=\alpha)$を取り出す}
            \Else
             \State 訪問回数$N(s)$から上位$k$の行動${a_0, a_1, ..., a_{k-1}}(=\alpha)$を取り出す
            \EndIf
           \For {each $a_i$ in $\alpha$}
             \State $s_{{next}_i} \gets T(s_{now}, a_i)$
             \State $\zeta(s_{now},s_{{next}_i})(={s_{now}, a_i, s_{{next}_i}})$を$Z$の末尾に追加
           \EndFor
           \If{l=1}
             \Return $Z$
           \EndIf
           \State $i \gets 1$
           \While {$i < l$}
                \For {each $\zeta(s_{now}, s_{j})$ in $Z$}
                    \State $\zeta(s_{now}, s_{j})$ を $Z$からポップ
                    \If{$s$終了状態のとき}
                        \State $\zeta(s_{now}, s_{j})$ を$k$回$Z$の末尾に追加
                        \State continue
                    \EndIf
                    \If{$s$が未探索のとき}
                       \State \underline{方策$P(s)$から上位$k$の行動${a_0, a_1, ..., a_{k-1}}(=\alpha)$を取り出す}
                   \Else
                    \State 訪問回数$N(s)$から上位$k$の行動${a_0, a_1, ..., a_{k-1}}(=\alpha)$を取り出す
                   \EndIf
                    
                    \For {each $a_i$ in $\alpha$}
                        \State $s_{{next}_j} \gets T(s_{j}, a_i)$
                        \State $\zeta(s_{now},s_{{next}_i})(=\zeta(s_{now}, s_{j}).append({a_i, s_{{next}_i}}))$を$Z$の末尾に追加
                    \EndFor
                    
                \EndFor     
           \EndWhile
           \Return $Z(={\zeta(s_{start}, {s'}_0), ..., \zeta(s_{start}, {s'}_{k^l-1})})$
        \EndFunction
        \Function {Traverse}{$s, \zeta(s_{start}, s)$}
        \State $s_{now} \gets s$
        \State $\zeta_r \gets \zeta(s_{start}, s)$
        \While {$s_{now}$が終了状態でない}
            \If{$s$が未探索のとき}
                \State \underline{$a_t \gets \textrm{argmax}_a P(s_{now}, a)$}
            \Else
                \State $a_t \gets \textrm{argmax}_a N(s_{now}, a)$
            \EndIf
            \State $s_n \gets T(s_{now}, a_t)$
            \State $\zeta_r.append({a_t, s_n})$
            \State $s_{now} \gets s_n$
        \EndWhile
        \Return $\zeta_r$
        \EndFunction
       
        
    \end{algorithmic}
\end{algorithm}
\section{システム実験における提案手法の変更}
システム実験では画面の左下に盤面$s$の訪問回数$N(s)$と局面評価$V(s, a)$を表示する。このとき
$V(s, a)$の絶対値(どちらのプレイヤーの勝利に近いか)と予想図の勝敗が異なる場合、ユーザーの混乱を招く可能性がある。
そのため、提案手法では勝敗が$V(s, a)$の絶対値と一致する軌道を表示するように修正を施した。
システム実験用に修正された疑似コードは以下となる。(下線が変更部分)
\begin{algorithm}
    \caption{提案手法のアルゴリズム(ニューロ補間あり)}
    \begin{algorithmic}[1]       
        
        \Function {Traverse}{$s, \zeta(s_{start}, s)$}
        \State  \underline{$\zeta(s_{start}, s)$から$s_{start}$の次の行動$a$を取り出す}
        \State \underline{$v \gets V(s, a)$}
        \State $s_{now} \gets s$
        \State $\zeta_r \gets \zeta(s_{start}, s)$
        \While {$s_{now}$が終了状態でない}
            \If{$s$が未探索のとき}
                \State \underline{$a_t \gets \textrm{argmax}_a P(s_{now}, a)$}
            \Else
                \State $a_t \gets \textrm{argmax}_a N(s_{now}, a)$
            \EndIf
            \State $s_n \gets T(s_{now}, a_t)$
            \State $\zeta_r.append({a_t, s_n})$
            \State $s_{now} \gets s_n$
        \EndWhile
        \If{\underline{$v$と$V(s_{now})$の絶対値が異なるとき}}
           \Return null
        \EndIf
        \Return $\zeta_r$
        \EndFunction
       
        
    \end{algorithmic}
\end{algorithm}
