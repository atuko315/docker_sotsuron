\chapter{評価実験}

\label{chap:evaluation}

\section{実験条件の設定}

\subsection{データセット}
為

\subsection{比較手法}
一

\section{あああの予測}

\subsection{実験方法}
2017年

\subsection{実験結果}
表\ref{table:result-1}に

\begin{comment}
	また,ああああああ
\end{comment}

\begin{table}[H]
	\caption{あああといいいの予測誤差}
	\centering
	\scalebox{0.98}[0.98]{
		\begin{tabular}{c|c|c|c|c|c|c}
			\multicolumn{1}{c}{} & \multicolumn{2}{|c|}{2019} 
			& \multicolumn{2}{c|}{2018} & \multicolumn{2}{c}{2017}\\ \hline \hline
			モデル    & ああ & いい & ああ & いい & ああ & いい \\ \hline
			Naive    & \bf{1} & 1 & \bf{1} & 1 & \bf{1} & 1 \\
			TCN      & 1.0895 & 0.9032 & 1.4791 & \bf{0.9198} & 1.2888 & 0.8555 \\
			LSTM     & 1.0384 & 0.9295 & 1.4917 & 0.9725 & 1.1627 & 0.8541 \\
			提案手法  & 1.0977 & \bf{0.8698} & 1.3824 & 0.9439 & 1.2061 & \bf{0.8516} \\
		\end{tabular}
	}
	\label{table:result-1}
\end{table}
